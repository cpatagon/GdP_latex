%!TEX root = DisenoProyecto_LuisGomez.tex
% SUPUESTOS 
%\begin{consigna}{red}
%``Para el desarrollo del presente proyecto se supone que: ...''
%
%\begin{itemize}
%	\item Supuesto 1
%	\item Supuesto 2...
%\end{itemize}
%
%Por ejemplo, se podrían incluir supuestos respecto a disponibilidad de tiempo y recursos humanos y materiales, sobre la factibilidad técnica de distintos aspectos del proyecto, sobre otras cuestiones que sean necesarias para el éxito del proyecto como condiciones macroeconómicas o reglamentarias.
%\end{consigna}
Mediante la implementación simultánea de tres sensores ópticos dedicados a la cuantificación de partículas finas atmosféricas (MP2,5), se espera generar conjunto de datos replicados que permitan realizar análisis estadísticos en tiempo real. Este diseño metodológico no solamente facilita la obtención de valores promedio más confiables, sino que también posibilita un proceso de validación o exclusión de observaciones anómalas que se desvíen significativamente de la media poblacional. Desde una perspectiva estadística, se espera que un mayor volumen muestral contribuirá al incremento tanto de la precisión como de la exactitud en las mediciones. Concretamente, se espera lograr una reducción en el Error Estándar de la Media (ESM) y un acercamiento más preciso al valor verdadero de la media poblacional ($\mu$), conforme a los principios del Teorema del Límite Central.

En términos de robustez del sistema, la presencia de múltiples sensores se estima que proporcione una capa adicional de confiabilidad. En caso de mal funcionamiento de algún sensor, el sistema de microcontrolador estará diseñado para detectar el problema, lo que permite la implementación de soluciones preventivas antes de una posible interrupción completa del equipo.

Finalmente, aunque estos sensores ópticos no están actualmente reconocidos por la Agencia de Protección Ambiental (EPA) como una técnica analítica estándar para la medición de MP2,5, su adopción por parte de autoridades y entidades gubernamentales ayudaría como un complemento eficaz y económico a las redes de monitoreo existentes. Esta incorporación no solo fortalecería las estrategias de monitoreo actualmente en uso, sino que también podría servir como un primer paso para establecer estaciones de monitoreo en áreas que actualmente carecen de ellas.


\subsection{Otros Supuestos}

\begin{itemize}
	
	\item \textbf{Calibración con estaciones de referencia:} se espera que los sensores puedan calibrarse utilizando estaciones de referencia EPA certificadas por instituciones con competencia.
	
	\item \textbf{Precisión de datos históricos:} se asume que los datos obtenidos durante los procesos de calibración con sistemas gravimétricos o estaciones de monitoreo existentes son precisos y representativos. Además, se espera que estos datos sean comparables a pesar de las diferencias en las frecuencias de muestreo entre las tecnologías.
	
	\item \textbf{Mantenimiento de la calibración:} una vez calibrados, se espera que los sensores mantengan su calibración durante todo el periodo de recolección de datos, sin requerir ajustes frecuentes para mantener su precisión y exactitud.
	
	\item \textbf{Estabilidad de condiciones ambientales:} se asume que las condiciones ambientales, como temperatura y humedad, no afectarán significativamente la precisión y la capacidad de medición de los sensores durante periodos prolongados.
	
	\item \textbf{Ausencia de interferencias:} se presupone que otras partículas o sustancias en el ambiente, como los aerosoles de agua, no interferirán significativamente en la medición del material particulado de interés.
	
	\item \textbf{Energía y conectividad constantes:} se espera contar con un suministro constante y fiable de energía y conectividad de datos durante la ejecución del proyecto.
	
	\item \textbf{Aceptación por parte de los interesados:} se anticipa que autoridades ambientales y otros grupos de interés estarán abiertos a considerar y, posiblemente, adoptar esta tecnología si se demuestra su eficacia.
	
	\item \textbf{Escalabilidad del sistema:}  se prevé que el sistema podrá funcionar en exteriores y escalarse para cubrir áreas geográficas más grandes o para incorporar tipos de mediciones ambientales adicionales sin cambios significativos en su arquitectura.
	
	\item \textbf{Futura conformidad regulatoria:} aunque los sensores no están actualmente estandarizados por la EPA, se espera que futuras regulaciones puedan incluir este tipo de tecnologías. Es relevante señalar que la Comunidad Económica Europea está evaluando actualmente esta posibilidad.
	
	\item \textbf{Participación comunitaria:} si los datos recolectados se utilizaran para la toma de decisiones a nivel comunitario, se espera un nivel adecuado de compromiso y participación por parte de la comunidad local.
	
	\item \textbf{Costos operativos controlables:} se estima que los costos operativos y de mantenimiento del sistema se mantendrán dentro de un rango predecible y manejable durante su vida útil.
	
\end{itemize}


