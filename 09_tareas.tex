%!TEX root = DisenoProyecto_LuisGomez.tex
%\begin{consigna}{red}
%El WBS debe tener relación directa o indirecta con los requerimientos.  Son todas las actividades que se harán en el proyecto para dar cumplimiento a los requerimientos. Se recomienda mostrar el WBS mediante una lista indexada:
%
%\begin{enumerate}
%\item Grupo de tareas 1
%	\begin{enumerate}
%	\item Tarea 1 (tantas h)
%	\item Tarea 2 (tantas hs)
%	\item Tarea 3 (tantas h)
%	\end{enumerate}
%\item Grupo de tareas 2
%	\begin{enumerate}
%	\item Tarea 1 (tantas h)
%	\item Tarea 2 (tantas h)
%	\item Tarea 3 (tantas h)
%	\end{enumerate}
%\item Grupo de tareas 3
%	\begin{enumerate}
%	\item Tarea 1 (tantas h)
%	\item Tarea 2 (tantas h)
%	\item Tarea 3 (tantas h)
%	\item Tarea 4 (tantas h)
%	\item Tarea 5 (tantas h)
%	\end{enumerate}
%\end{enumerate}
%
%Cantidad total de horas: (tantas h)
%
%Se recomienda que no haya ninguna tarea que lleve más de 40 h. 
%
%\end{consigna}


\label{sec:wbs}

En el siguiente apartado, se llevará a cabo un desglose detallado de las actividades y tareas propuestas para la ejecución del proyecto de tesis de la Carrera. Cada actividad será identificada utilizando la numeración del work breakdown structure (WBS) para asegurar una organización y seguimiento efectivos. Adicionalmente, se especificarán tanto los tiempos parciales como los tiempos totales requeridos para cada tarea, expresados en formato de hora.

Esta estructuración tiene como objetivo proporcionar una visión completa y ordenada del proyecto, permitiendo una asignación y gestión eficiente de los recursos temporales. Como se muestra a continuación, se estima el trabajo total del proyecto de tesis en unas 730 horas cronológicas, divididas en ocho secciones.



\begin{enumerate}
	\item\textbf{ Gestión del proyecto (100 h)}
	\begin{enumerate}
		\item Generación de requerimientos (20 h)
		\item Planificación del proyecto (40 h)
		\item Planificación de entregables (10 h)
		\item Revisión y ajustes de planificación (30 h)
	\end{enumerate}
	
	\item\textbf{Diseño general (60 h)}
	\begin{enumerate}
		\item Ajuste al diseño conceptual (10 h)
		\item Diagramas de flujo y arquitectura (20 h)
		\item Revisión y pruebas de diseño (30 h)
	\end{enumerate}
	
	\item \textbf{Construcción del hardware (120 h)}
	\begin{enumerate}
		\item Selección de componentes (20 h)
		\item Diseño de circuitos (40 h)
		\item Montaje y soldadura (30 h)
		\item Pruebas iniciales (30 h)
	\end{enumerate}
	
	\item \textbf{Diseño del firmware (110 h)}
	\begin{enumerate}
		\item Diseño de la arquitectura del software (20 h)
		\item Implementación de la medición de MP2,5 (30 h)
		\item Implementación de almacenamiento y comunicación (30 h)
		\item Implementación de funciones auxiliares (30 h)
	\end{enumerate}
	
	\item \textbf{Realización de pruebas (80 h)}
	\begin{enumerate}
		\item Diseño de casos de prueba (20 h)
		\item Ejecución de pruebas (40 h)
		\item Análisis de resultados (20 h)
	\end{enumerate}
	
	\item \textbf{Ajustes finales instrumento (40 h)}
	\begin{enumerate}
		\item Depuración de errores (20 h)
		\item Ajustes de performance (20 h)
	\end{enumerate}
	
	\item \textbf{Escritura de memoria y manuales (180 h)}
	\begin{enumerate}
		\item Marco teórico (40 h)
		\item Metodología (30 h)
		\item Implementación (20 h)
		\item Resultados y conclusiones (20 h)
		\item Introducción, resumen y otros (30 h)
		\item Manual de usuario (20 h)
		\item Manual técnico (20 h)
	\end{enumerate}
	
	\item \textbf{Entregas del trabajo final y cierre (40 h)}
	\begin{enumerate}
		\item Proceso de cierre (20 h)
		\item Entrega y presentación del trabajo final (20 h)
	\end{enumerate}
\end{enumerate}


Cantidad total de horas: 730 hs


