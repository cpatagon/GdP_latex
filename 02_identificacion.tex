%!TEX root = DisenoProyecto_LuisGomez.tex

%\begin{consigna}{red} % este comando se debe borrar para la entrega, junto con la contraparte \end{consigna}{red} 
% 
%\textbf{Nota importante:} borrar esto y todas las consignas en color rojo antes de entregar este documento). Esto se hace eliminando el par de comandos que forman el bloque consigna, \verb!\begin{consigna}{red}! y \verb!\end{consigna}{red}! del código. 
% 
%Es inusual que una misma persona esté en más de un rol, incluso en proyectos chicos. Si se considera que una persona cumple dos o más roles, entonces solo dejarla en el rol más importante. 
%
%Por ejemplo, si una persona es Cliente pero también colabora u orienta, dejarla solo como Cliente. Si una persona es el Responsable, no debe ser colocado también como miembro del equipo.


\begin{table}[ht]
\caption{Identificación de los interesados.}
\label{tab:interesados}
\begin{tabularx}{\linewidth}{@{}|l|X|X|l|@{}}
\hline
\rowcolor[HTML]{C0C0C0} 
Rol           & Nombre y Apellido & Organización 	& Puesto 	\\ \hline
%Auspiciante   &      -             &        -      	&    -    	\\ \hline
Cliente       & \clientename      &\empclientename	& Investigadora       	\\ \hline
%Impulsor      &        -           &       -       	&    -    	\\ \hline
Responsable   & \authorname       & FIUBA        	& Alumno 	\\ \hline
%Colaboradores &       -            &       -       	&     -   	\\ \hline
Orientador    & \supname	      & \pertesupname 	& Director Trabajo final \\ \hline
%Equipo        &      -       	  &           -   	&    -   	\\ \hline
%Opositores    &       -            &        -      	&       - 	\\ \hline
Usuario final &      Comunidades afectadas  &     -         	&      -  	\\ \hline
\end{tabularx}
\end{table}

%El Director suele ser uno de los Orientadores.
%
%No dejar celdas vacías; si no hay nada que poner en una celda colocar un signo ``-''.
%
%No dejar filas vacías; si no hay nada que poner en una fila entonces eliminarla.
%
%Es deseable listar a continuación las principales características de cada interesado.
% 
%Por ejemplo:
\begin{description}
\item [Cliente:] doctora en Química Ambiental especializada en monitoreo de calidad del aire. Actualmente lidera proyectos de relevancia nacional en Chile, como Fodequip Mayor y Fodecyt, enfocados en la implementación y evaluación de sensores de bajo costo para el monitoreo ambiental. Se espera su aporte en cuanto a definir características y requerimientos necesarios para un instrumental orientado a estimar las concentraciones de MP2,5.
%	\item Auspiciante: es riguroso y exigente con la rendición de gastos. Tener mucho cuidado con esto.
%	\item Equipo: Juan Perez, suele pedir licencia porque tiene un familiar con una enfermedad. Planificar considerando esto.
\item [Orientador:] ingeniero electrónico con amplia experiencia tanto en el ámbito profesional como en la docencia. Su contribución será fundamental en el diseño de la placa electrónica que soportará el instrumental y en la optimización de la programación del microcontrolador.

\item [Usuario final:] el usuario final está constituido principalmente por la población urbana expuesta a episodios de contaminación atmosférica relacionados con MP2,5. Estos episodios son especialmente prevalentes durante ciertos días de invierno y pueden representar un riesgo significativo para la salud de grupos vulnerables, como niños, ancianos y personas con enfermedades preexistentes.
\end{description}
%
%\end{consigna} % este comando se debe borrar para la entrega, junto con la contraparte \begin{consigna}{red}