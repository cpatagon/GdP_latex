%!TEX root = DisenoProyecto_LuisGomez.tex

%\begin{consigna}{red}
%Los requerimientos deben numerarse y de ser posible estar agruparlos por afinidad, por ejemplo:
%
%\begin{enumerate}
%	\item Requerimientos funcionales
%		\begin{enumerate}
%			\item El sistema debe...
%			\item Tal componente debe...
%			\item El usuario debe poder...
%		\end{enumerate}
%	\item Requerimientos de documentación
%		\begin{enumerate}
%			\item Requerimiento 1
%			\item Requerimiento 2 (prioridad menor)
%		\end{enumerate}
%	\item Requerimiento de testing...
%	\item Requerimientos de la interfaz...
%	\item Requerimientos interoperabilidad...
%	\item etc...
%\end{enumerate}
%
%Leyendo los requerimientos se debe poder interpretar cómo será el proyecto y su funcionalidad.
%
%Indicar claramente cuál es la prioridad entre los distintos requerimientos y si hay requerimientos opcionales. 
%
%No olvidarse de que los requerimientos incluyen a las regulaciones y normas vigentes!!!
%
%Y al escribirlos seguir las siguientes reglas:
%\begin{itemize}
%	\item Ser breve y conciso (nadie lee cosas largas). 
%	\item Ser específico: no dejar lugar a confusiones.
%	\item Expresar los requerimientos en términos que sean cuantificables y medibles.
%\end{itemize}
%
%\end{consigna}
En esta sección, se enumeran los requisitos del sistema de acuerdo a la experiencia del equipo ejecutor, características de la competencia y conversaciones con potenciales clientes. Los requisitos se clasifican en distintas categorías: requerimientos funcionales, requerimientos de hardware, requerimientos de software, requerimientos de interfaz de usuario, requerimientos de alimentación, requerimientos de evaluación y requerimientos de documentación. A cada requisito se le asigna una etiqueta de prioridad, como se muestra a continuación:
\begin{description}
	\item [\textbf{P1}]: \textbf{obligatorio}, este requisito es crucial y debe implementarse tal como se describe.
	\item [\textbf{P2}]: \textbf{importante}, es necesario implementar este requisito, aunque se pueden considerar alternativas razonables.
	\item [\textbf{P3}]: \textbf{recomendado}, su implementación es deseable, pero no esencial, y su omisión debe ser justificada.
	\item [\textbf{P4}]: \textbf{opcional}, este es un requisito que puede ser implementado a discreción del equipo de desarrollo.
\end{description}

\subsection{Requerimientos funcionales}
\begin{enumerate}[label=\alph*)]
	\item El instrumento debe incorporar al menos tres sensores dedicados a la medición de MP2,5. Prioridad: [\textbf{P1}].
	\item El dispositivo debe almacenar localmente los datos de MP2,5, ofreciendo la opción de consultar y vaciar la memoria cuando sea necesario. Prioridad: [\textbf{P1}]
	\item El sistema debe calcular automáticamente los parámetros estadísticos relevantes, como el promedio, la desviación estándar, valores mínimos y máximos, y detectar valores fuera de rango. Prioridad: [\textbf{P2}]
	\item La comunicación entre el nodo sensor y el servidor debe ser flexible, admitiendo conexiones cableadas o inalámbricas mediante protocolos como Wi-Fi (IEEE 802.11n) o LoRa. Prioridad: [\textbf{P3}]
	\item Cada instrumento debe ser identificable de manera única dentro del sistema, permitiendo garantizar una correcta asociación de los datos recolectados con el sensor. Prioridad: [\textbf{P2}]
	\item Todas las mediciones deben ser acompañadas de una estampa temporal proporcionada por un Reloj de Tiempo Real (RTC). Prioridad: [\textbf{P1}]
	\item El instrumento debe ofrecer funcionalidades de monitoreo remoto que permitan verificar su estado operativo, calibraciones y ajustes. Prioridad: [\textbf{P4}]
	\item El sistema debe ser capaz de generar y almacenar promedios temporales de las mediciones, tanto en intervalos de 60 minutos como en periodos de 24 horas. Prioridad: [\textbf{P2}]
\end{enumerate}
\subsection{Requerimientos de hardware}
\begin{enumerate}[label=\alph*)]
	\item Se usará una placa de desarrollo compatible con algún sistema de comunicación inalámbrica como Wi-Fi o LoRa. Prioridad: [\textbf{P3}]
	\item La placa de desarrollo deberá permitir conectar múltiples sensores para medir MP2,5, administrando y calculado los datos provenientes tanto de los sensores como el RTC. Prioridad: [\textbf{P1}]
	\item La comunicación entre los sensores y la placa será mediante protocolo I2C u otro similar que esté de acuerdo a las características del sensor y la placa. Prioridad: [\textbf{P1}]
	\item Rango de operación mínima de concentración del MP2,5: 0 a 500 $\mu g/m^3$ y precisión de medición: ±10\%. Prioridad: [\textbf{P2}]
\end{enumerate}
\subsection{Requerimientos de software}
\begin{enumerate}[label=\alph*)]
	\item Las variables en la que se administrarán los datos de MP2,5 serán del tipo ``constante punto flotante". Por lo tanto, el almacenamiento y las operaciones como promedio, mínimo, máximo u otro, deben también ser compatibles con este tipo de variable. Prioridad: [\textbf{P1}]
	\item Debido a que se utilizará un servidor remoto para almacenar y visualizar los datos, deberán implementarse los protocolos de comunicación acordes con el sistema de transmisión seleccionado. Prioridad: [\textbf{P4}]
	\item El programa deberá generar alarmas de funcionamiento y de fallas, que permitan al operador identificar el estado de operación del instrumento. Prioridad: [\textbf{P3}]
\end{enumerate}

\subsection{Requerimientos de interfaz de usuario}
\begin{enumerate}[label=\alph*)]
	\item El usuario debe poder acceder a los datos históricos medidos por el instrumento, ya sea leyendo la memoria incorporada en el instrumento o revisando los registros en el servidor. Prioridad: [\textbf{P2}]
	\item Debe generar alertas y notificaciones basadas en umbrales predefinidos de concentración de MP2,5. Por ejemplo, niveles críticos de medición, valores fuera de rango de medición, etc. Prioridad: [\textbf{P4}]
	\item Deberá permitir poner el sistema en modo de ahorro de energía, cuando exista una desconexión del sistema eléctrico. Prioridad: [\textbf{P3}]
\end{enumerate}
\subsection{Requerimientos de alimentación}
\begin{enumerate}[label=\alph*)]
	\item El instrumento contará con una fuente de energía compatible con la red doméstica de 220 V. Prioridad: [\textbf{P1}]
	\item El servidor y otros componentes anexos al instrumento estarán alimentados principalmente con 220 V mediante tomacorriente. Prioridad: [\textbf{P2}]
	\item A modo de seguridad, el instrumento podrá funcionar en un modo de ahorro y por tiempo reducido, mediante una batería recargable de al menos 2000 mAh. Prioridad: [\textbf{P4}]
\end{enumerate}
\subsection{Requerimientos de gabinete}
\begin{enumerate}[label=\alph*)]
	\item Los componentes del instrumento deben estar dispuestos en un gabinete individual de material plástico, con acceso a alimentación eléctrica y la entrada y salida de aire hacia y desde el censor de MP2,5. Prioridad: [\textbf{P1}]
	\item El gabinete debe ser estanco con clasificación acorde a IP65 o superior. Es decir, el gabinete debe ser adecuado para su uso en exteriores, con exposición al polvo y al agua en forma de lluvia. Prioridad: [\textbf{P2}]
	\item Facilidades para que el equipo pueda ser montado sobre postes, paredes o techos. [\textbf{P3}]
\end{enumerate}
\subsection{Requerimientos de evaluación}
\begin{enumerate}[label=\alph*)]
	\item Se debe probar el sistema en diversas condiciones atmosféricas, como humedad, temperatura, lluvias, etc. Prioridad:[\textbf{P2}]
	\item La conexión inalámbrica debe tener un área de cobertura mínima de 0,2 km. Prioridad:[\textbf{P4}]
	\item Se deben realizar pruebas de calibración con censores certificados, antes de la implementación completa. Prioridad: [\textbf{P1}]
\end{enumerate}
\subsection{Requerimientos de documentación}
\begin{enumerate}[label=\alph*)]
	\item Elaborar un manual con las características principales del instrumento, mantenimiento y sus limitaciones. Prioridad:[\textbf{P1}]
	\item La generación de tablas con posibles fallas, códigos de error y soluciones. Prioridad:[\textbf{P3}]
	\item Esquemáticos con la distribución de componentes, conexiones y alimentación eléctrica del instrumento. Prioridad: [\textbf{P1}]
\end{enumerate}
