%!TEX root = DisenoProyecto_LuisGomez.tex
%\begin{consigna}{red}
%Elija al menos diez requerientos que a su criterio sean los más importantes/críticos/que aportan más valor y para cada uno de ellos indique las acciones de verificación y validación que permitan asegurar su cumplimiento.
%
%\begin{itemize} 
%\item Req \#1: copiar acá el requerimiento.
%
%\begin{itemize}
%	\item Verificación para confirmar si se cumplió con lo requerido antes de mostrar el sistema al cliente. Detallar 
%	\item Validación con el cliente para confirmar que está de acuerdo en que se cumplió con lo requerido. Detallar  
%\end{itemize}
%
%\end{itemize}
%
%Tener en cuenta que en este contexto se pueden mencionar simulaciones, cálculos, revisión de hojas de datos, consulta con expertos, mediciones, etc.  Las acciones de verificación suelen considerar al entregable como ``caja blanca'', es decir se conoce en profundidad su funcionamiento interno.  En cambio, las acciones de validación suelen considerar al entregable como ``caja negra'', es decir, que no se conocen los detalles de su funcionamiento interno.
%
%\end{consigna}






Elija al menos diez requerimientos que a su criterio sean los más importantes/críticos/que aportan más valor y para cada uno de ellos indique las acciones de verificación y validación que permitan asegurar su cumplimiento.

\begin{itemize} 
\item Req \#1: Funcionamiento correcto y preciso de los sensores para medir la calidad del agua.

\begin{itemize}
	\item Verificación: Realizar pruebas y simulaciones para asegurarse de que los sensores miden con precisión los parámetros de la calidad del agua. Consultar con expertos y revisar las hojas de datos de los sensores para confirmar la precisión y fiabilidad de las mediciones.
	\item Validación: Realizar mediciones en campo y comparar los resultados obtenidos con mediciones estándar y aceptadas para asegurar la exactitud y conformidad de los sensores en condiciones reales.
\end{itemize}

\item Req \#2: Transmisión de datos segura y sin fallos desde los sensores hasta la base de datos.

\begin{itemize}
	\item Verificación: Probar la transmisión de datos en diferentes condiciones y con diferentes volúmenes de datos para asegurar la robustez y fiabilidad del sistema de transmisión.
	\item Validación: Confirmar con el cliente mediante pruebas de transmisión de datos en tiempo real para asegurar que los datos se reciben correctamente y en tiempo.
\end{itemize}

\item Req \#3: Sistema de alimentación energética ininterrumpida y fiable.

\begin{itemize}
	\item Verificación: Realizar pruebas de estrés al sistema de alimentación para asegurar su capacidad de mantenerse operativo en situaciones de corte de energía.
	\item Validación: Demostrar al cliente mediante pruebas de funcionamiento prolongado que el sistema puede operar de forma continua sin interrupciones debidas a fallos de energía.
\end{itemize}



\end{itemize}

