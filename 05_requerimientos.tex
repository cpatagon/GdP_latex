%\begin{consigna}{red}
%Los requerimientos deben numerarse y de ser posible estar agruparlos por afinidad, por ejemplo:
%
%\begin{enumerate}
%	\item Requerimientos funcionales
%		\begin{enumerate}
%			\item El sistema debe...
%			\item Tal componente debe...
%			\item El usuario debe poder...
%		\end{enumerate}
%	\item Requerimientos de documentación
%		\begin{enumerate}
%			\item Requerimiento 1
%			\item Requerimiento 2 (prioridad menor)
%		\end{enumerate}
%	\item Requerimiento de testing...
%	\item Requerimientos de la interfaz...
%	\item Requerimientos interoperabilidad...
%	\item etc...
%\end{enumerate}
%
%Leyendo los requerimientos se debe poder interpretar cómo será el proyecto y su funcionalidad.
%
%Indicar claramente cuál es la prioridad entre los distintos requerimientos y si hay requerimientos opcionales. 
%
%No olvidarse de que los requerimientos incluyen a las regulaciones y normas vigentes!!!
%
%Y al escribirlos seguir las siguientes reglas:
%\begin{itemize}
%	\item Ser breve y conciso (nadie lee cosas largas). 
%	\item Ser específico: no dejar lugar a confusiones.
%	\item Expresar los requerimientos en términos que sean cuantificables y medibles.
%\end{itemize}
%
%\end{consigna}