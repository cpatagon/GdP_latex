%!TEX root = DisenoProyecto_LuisGomez.tex
%\begin{consigna}{red}
%Establecer las pautas de trabajo para realizar una reunión final de evaluación del proyecto, tal que contemple las siguientes actividades:
%
%\begin{itemize}
%	\item Pautas de trabajo que se seguirán para analizar si se respetó el Plan de Proyecto original:
%	 - Indicar quién se ocupará de hacer esto y cuál será el procedimiento a aplicar. 
%	\item Identificación de las técnicas y procedimientos útiles e inútiles que se emplearon, y los problemas que surgieron y cómo se solucionaron:
%	 - Indicar quién se ocupará de hacer esto y cuál será el procedimiento para dejar registro.
%	\item Indicar quién organizará el acto de agradecimiento a todos los interesados, y en especial al equipo de trabajo y colaboradores:
%	  - Indicar esto y quién financiará los gastos correspondientes.
%\end{itemize}
%
%\end{consigna}


Se llevará a cabo una reunión final con el propósito de evaluar el proyecto en su totalidad. Durante esta reunión, se examinará de manera exhaustiva el grado de cumplimiento del Plan de Proyecto inicial y se identificarán las técnicas y procedimientos que resultaron ser útiles, así como aquellos que no lo fueron. Asimismo, se analizarán los problemas que hayan surgido durante la implementación del proyecto y las soluciones que se hayan adoptado.

Se llevará a cabo un acto de cierre para expresar nuestro agradecimiento a la colaboración de todos los participantes, como cierre final de la titulación. Toda la documentación generada, incluyendo escritos, documentos que describen el proceso y evaluaciones, será compendiada como parte del documento de tesis final. Además, todos estos antecedentes estarán debidamente archivados y disponibles en el sitio web del postgrado, accesibles a través del siguiente \href{https://lse.posgrados.fi.uba.ar/trabajo-final/archivo-historico/consulta-de-trabajos-finales-listado-completo}{enlace en internet}.

\subsection{Pautas de trabajo para analizar el respeto al Plan de Proyecto original}
\begin{itemize}
	\item \textbf{Responsable:} Luis Gómez (jefe de proyecto).
	\item \textbf{Procedimiento:} comparar los resultados obtenidos con los objetivos establecidos en el Plan de Proyecto original, documentar desviaciones y analizar sus causas.
	\item \textbf{Registro:} texto que contiene análisis comparativo.
\end{itemize}

\subsection{Identificación de técnicas y procedimientos y solución de problemas}
\begin{itemize}
	\item \textbf{Responsable:} Luis Gómez (jefe de proyecto).
	\item \textbf{Procedimiento:} celebrar una reunión retrospectiva para discutir y documentar las lecciones aprendidas y los problemas surgidos.
	\item \textbf{Registro:} texto que contiene ``lecciones aprendidas".
\end{itemize}

\subsection{Acto de agradecimiento}
\begin{itemize}
	\item \textbf{Organizador:} Luis Gómez (jefe de proyecto).
	\item \textbf{Procedimiento:} ceremonia de agradecimiento y entrega de certificados de reconocimiento. En este acto se agradecerá a todas las personas
	que contribuyeron, jurados, docentes, autoridades de la carrera de especialización,
	colegas y autoridades.
\end{itemize}

\subsection{Cronograma de cierre}
\begin{itemize}
	\item \textbf{Reunión de evaluación del proyecto:} 15 de mayo del 2024
	\item \textbf{Entrega de documento con análisis comparativo:} 22 de mayo del 2024
	\item \textbf{Entrega de documento con lecciones aprendidas:} 22 de mayo del 2024
	\item \textbf{Ceremonia de agradecimiento:} 22 de mayo del 2024
\end{itemize}

\subsection*{Participantes}
Todos los miembros del equipo del proyecto, colaboradores, interesados y patrocinadores están invitados a participar en las actividades de cierre.
